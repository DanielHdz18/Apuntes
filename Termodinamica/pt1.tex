\documentclass[master.tex]{subfiles}

\begin{document}
\section{Conceptos y propiedades termodinámicas}

\subsection{Origen y alcance de la termodinámica}

    La termodinámica nacio en el siglo XIX, debido al surgimiento de las maquinas de vapor y la necesidad de estudiar su comportamiento. La termodinámica se basa esencialmente en dos postulados o leyes fundamentales, que resumen los resultados experimentales de la interconversión de las diferentes formas de energía. Estos postulados o leyes no tiene demostración en el sentido matématico y no se pueden deducir de ningun otro conjunto de leyes físicas, se consideran validas ya que se ajustan a los resultados experimentales y a la falta de contraejemplos, es decir jamas se ha encontrando que algún proceso de interconversión de energía viole estos principios \parencites{glasstone}{smith-vanness}. 
    
    Todos los procesos de interconversión de energía (o procesos termodinámicos) requieren de un equipo para este fin, los ingenieros utilizan las leyes de la termodinámica en combinación con otras ciencias como la mécanica de fluids y los fenómenos de transporte para analizar, diseñar y optimizar estos equipos donde se transforma la energía de un tipo a otra. El resultado de la aplicación de la termodinámica en estos equipos puede conllevar una mejora en el rendimiento de estos, a la reducción de los costos totales, a un menor uso de recursos escasos o a un menor impacto ambiental \parencites{wark}{moranshapiro}.
    
    Ejemplo de interconversiones cotidianas y los equipos utilizados son la conversión de calor en electricidad en las centrales electricas para la generación de energía eléctrica, de trabajo eléctrico en un efecto de enfriamiento para el enfriamiento en los aires acondicionados, de trabajo mecánico en energía cinética en los motores de los automóviles \parencite{faires}.

    Las propiedades termodinámicas, que seran definidas más adelante, y las relaciones entre los distintos tipos de energía se pueden estudiar por dos métodos. El primero de ellos se conoce como \textbf{termodinámica clásica}, en este modelo las propiedades y leyes son independientes de la estructura atómica y molecular así como independientes de los mecanismos de reacción y de la interacción entre las partículas. Este es un punto de vista macroscópico de la materia y debido a que se basa en medidas macroscópicas a medida que se avanza en el conocimiento de la estructura atómica las leyes termodinámicas no cambian. \parencites{faires}{glasstone}{wark}.

    El segundo método es el que se denomina como \textbf{termodinámica estadística}, este se basa en el comportamiento de las moléculas y su estructura interna, es decir es el punto de vista microscópico de la materia y postula que los valores de las propiedades macroscópicas (como la presión, temperatura y la densidad) que pueden ser medidas directa o indirectamente son el reflejo de un promedio estadístico del comportamiento interno de las moléculas o partículas \parencites{faires}{wark}.

    Ambos métodos tienen sus ventajas y desventajas. El enfoque clásico es simple, intuitivo y requiere de matemáticas sencillas, sin embargo no permite la comprensión de la naturaleza de ciertos fenómenos de forma completa, por el contrario el punto de vista de la termodinámica estadística permite una mejor compresión de estos fenómenos pero a cambio se requiere de un conocimiento de matemáticas más avanzado y de la aceptación de un modelo atómico de la materia \parencite{faires}. 

    En este texto se usara la termodinámica clásica y cuando sea necesario se hara uso de los conceptos de la termodinámica estadística, es importante aclarar que ningun ley de la termodinámica clásica da información sobre el comportamiento a nivel microscópico de la materia, caso contrario a la termodinámica estadística que si puede dar información del comportamiento macroscópico de la materia \parencite{smith-vanness}.

    

\printbibliography[title=Bibliografía del capítulo]
    

\end{document}