\documentclass[master.tex]{subfiles}
\usepackage{polyglossia} %configuración de idioma
\setmainlanguage{spanish} %elegir idioma español
\usepackage{fontspec} %paquete para cargar fuentes
\setsansfont{cmunb}[Extension=.otf,
UprightFont=*mr,
ItalicFont=*mo,
BoldFont=*sr, % semibold
BoldItalicFont=*so, % semibold oblique
NFSSFamily=cmbr]
\usepackage{cmbright} %fuentes
\usepackage{microtype} %mejorar generales al documento pdf
\usepackage{amsmath} %paquete matemáticos
\usepackage{amsfonts} %fuentes matemáticas varias
\usepackage{amssymb} %símbolos matemáticos varios
\usepackage{graphicx} %incluir gráficos
\usepackage[left=2cm, right=2cm, top=2cm, bottom=2cm]{geometry} %paquete de configuración de margenes
\usepackage[dvipsnames]{xcolor} %paquete de colores
\usepackage[backend=biber, style=apa]{biblatex} %Bibliografía APA
\DeclareLanguageMapping{spanish}{spanish-apa} %Idioma Apa
%\addbibresource{Libros/Biblioteca.bib} %Archivo .bib
\usepackage[hidelinks]{hyperref}
\setlength{\bibitemsep}{0.3\baselineskip} %espaciado entre referencias
\setlength{\parindent}{0mm} %sangria de parrafo
\setlength{\parskip}{1.5ex} %espaciado de parrafo
\usepackage{float} %Agregar imágenes en cualquier lugar

\addbibresource{../Bibliografia/Libros.bib}

\begin{document}
    \section{Conceptos y propiedades termodinámicas}

    \subsection{Origen y alcance de la termodinámica}

    La materia de objeto de la termodinámica se basa esencialmente en dos postulados o principios fundamentales, que resumen los resultados experimentales de la interconversión de las diferentes formas de energía \parencite{glasstone}. Ejemplo de interconversiones cotidianas son la conversión de calor en electricidad en la generación de energía eléctrica, de trabajo eléctrico para el enfriamiento en los aires acondicionados, de trabajo mecánico en energía cinética en los automóviles \parencite{faires}.

    Las propiedades termodinámicas y las relaciones entre los distintos tipos de energía se pueden estudiar por dos métodos. El primero de ellos se conoce como \textbf{termodinámica clásica}, en este modelo las propiedades y leyes son independientes de la estructura atómica y molecular así como independientes de los mecanismos de reacción y de la interacción entre las partículas. Este es un punto de vista macroscópico de la materia y debido a que se basa en medidas macroscópicas a medida que se avanza en el conocimiento de la estructura atómica las leyes termodinámicas no cambian. \parencites{faires}{glasstone}{wark}

\end{document}