\documentclass[master.tex]{subfiles}


\begin{document}
    \section{Conceptos y propiedades termodinámicas}

    \subsection{Origen y alcance de la termodinámica}

    La materia de objeto de la termodinámica se basa esencialmente en dos postulados o principios fundamentales, que resumen los resultados experimentales de la interconversión de las diferentes formas de energía \parencite{glasstone}. Ejemplo de interconversiones cotidianas son la conversión de calor en electricidad en la generación de energía eléctrica, de trabajo eléctrico para el enfriamiento en los aires acondicionados, de trabajo mecánico en energía cinética en los automóviles \parencite{faires}.

    Las propiedades termodinámicas y las relaciones entre los distintos tipos de energía se pueden estudiar por dos métodos. El primero de ellos se conoce como \textbf{termodinámica clásica}, en este modelo las propiedades y leyes son independientes de la estructura atómica y molecular así como independientes de los mecanismos de reacción y de la interacción entre las partículas. Este es un punto de vista macroscópico de la materia y debido a que se basa en medidas macroscópicas a medida que se avanza en el conocimiento de la estructura atómica las leyes termodinámicas no cambian. \parencites{faires}{glasstone}{wark}.

    El segundo método es el que se denomina como \textbf{termodinámica estadística}, este se basa en el comportamiento de las moléculas y su estructura interna, es decir es el punto de vista microscópico de la materia y postula que los valores de las propiedades macroscópicas (como la presión, temperatura y la densidad) que pueden ser medidas directa o indirectamente son el reflejo de un promedio estadístico del comportamiento interno de las moléculas o partículas \parencites{faires}{wark}.

    Ambos métodos tienen sus ventajas y desventajas. El enfoque clásico es simple, intuitivo y requiere de matemáticas sencillas, sin embargo no permite la comprensión de la naturaleza de ciertos fenómenos de forma completa, por el contrario el punto de vista de la termodinámica estadística permite una mejor compresión de estos fenómenos pero a cambio se requiere de un conocimiento de matemáticas más avanzado y de la aceptación de un modelo atómico de la materia \parencite{faires}. 

    
    
    \printbibliography[title=Bibliografía del capítulo]
    

\end{document}