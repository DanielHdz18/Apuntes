\documentclass[../master.tex]{subfiles}

\begin{document}
\section{Conceptos y propiedades termodinámicas}

\subsection{Origen y alcance de la termodinámica}

La termodinámica nació en el siglo XIX, debido al surgimiento de las máquinas de vapor y la necesidad de estudiar su comportamiento\footnote{Una lectura más detallada sobre la historia de la termodinámica se puede encontrar en \cite{clavell}.}. La termodinámica se basa esencialmente en dos postulados o leyes fundamentales, que resumen los resultados experimentales de la interconversión de las diferentes formas de energía. Estos postulados o leyes no tienen demostración en el sentido matemático y no se pueden deducir de ningún otro conjunto de leyes físicas, se consideran válidas ya que se ajustan a los resultados experimentales y a la falta de contraejemplos, es decir jamás se ha encontrado que algún proceso de interconversión de energía viole estos principios \parencites{glasstone}{smith-vanness}.

Todos los procesos de interconversión de energía (o procesos termodinámicos) requieren de un equipo para este fin, los ingenieros utilizan las leyes de la termodinámica en combinación con otras ciencias como la mecánica de fluidos y los fenómenos de transporte para analizar, diseñar y optimizar estos equipos donde se transforma la energía de un tipo a otra. El resultado de la aplicación de la termodinámica en estos equipos puede conllevar una mejora en el rendimiento, a la reducción de los costos totales, a un menor uso de recursos escasos o a un menor impacto ambiental \parencites{wark}{moranshapiro}.

Ejemplow de interconversiones cotidianas y los equipos utilizados son la conversión de calor en electricidad en las centrales eléctricas para la generación de energía eléctrica, de trabajo eléctrico en un efecto de enfriamiento para el enfriamiento en los aires acondicionados, de trabajo mecánico en energía cinética en los motores de los automóviles \parencite{faires}.

Las propiedades termodinámicas, que serán definidas más adelante, y las relaciones entre los distintos tipos de energía se pueden estudiar por dos métodos. El primero de ellos se conoce como \textbf{termodinámica clásica}, en este modelo las propiedades y leyes son independientes de la estructura atómica y molecular así como independientes de los mecanismos de reacción y de la interacción entre las partículas. Este es un punto de vista macroscópico de la materia y debido a que se basa en medidas macroscópicas a medida que se avanza en el conocimiento de la estructura atómica las leyes termodinámicas no cambian \parencites{faires}{glasstone}{wark}.

El segundo método es el que se denomina como \textbf{termodinámica estadística}, este se basa en el comportamiento de las moléculas y su estructura interna, es decir es el punto de vista microscópico de la materia y postula que los valores de las propiedades macroscópicas (como la presión, temperatura y la densidad) que pueden ser medidas directa o indirectamente son el reflejo de un promedio estadístico del comportamiento interno de las moléculas o partículas \parencites{faires}{wark}.

Ambos métodos tienen sus ventajas y desventajas. El enfoque clásico es simple, intuitivo y requiere de matemáticas sencillas, sin embargo no permite la comprensión de la naturaleza de ciertos fenómenos de forma completa, por el contrario el punto de vista de la termodinámica estadística permite una mejor comprensión de estos fenómenos pero a cambio se requiere de un conocimiento de matemáticas más avanzado y de la aceptación de un modelo atómico de la materia \parencite{faires}.

En este texto se usara la termodinámica clásica y, cuando sea necesario se hará uso de los conceptos de la termodinámica estadística, es importante aclarar que ninguna ley de la termodinámica clásica da información sobre el comportamiento a nivel microscópico de la materia, caso contrario a la termodinámica estadística que si puede dar información del comportamiento macroscópico de la materia \parencite{smith-vanness}.

\subsection{Conceptos y propiedades fundamentales de la termodinámica}

\subsubsection{Sistemas de unidades}

Los conceptos de dimensión y unidad son comúnmente confundidos, para solventar esta confusión se presentan las definiciones de ambos conceptos. Las dimensiones son cantidades físicas mensurables, mientras que las unidades son patrones arbitrarios que se usan para realizar medidas sobre una dimensión, por ejemplo una dimensión es la longitud, mientras que el metro es un patrón de medida que permite describir una longitud de manera cuantitativa \parencites{rubenstein}{doran}.


La magnitud de cualquier cantidad física (o de cualquier dimensión) puede considerarse como el producto de un número por una unidad, donde la unidad adquiere una entidad matemática y puede considerarse como un número de tamaño variable. Por ejemplo la distancia entre dos puntos puede ser expresada como \qty{1}{\metre} o \qty{3.28}{\feet} o \qty{100}{\centi\metre}, en donde \numlist{1;3.28;100} son el número de unidades y \unit{m}, \unit{\feet}, \unit{\centi\metre} son el tamaño de cada unidad \parencite{coulson1}.

Existen determinadas reglas para poder operar con las unidades. La primera de ellas establece que solo es posible, sumar, restar o expresar una igualdad entre dos magnitudes físicas si estas tienen las mismas unidades (y por tanto las mismas dimensiones), sin embargo la segunda regla establece que es posible multiplicar o dividir unidades entre sí sin ninguna restricción, es por esta segunda regla que a partir de una serie de dimensiones, denominadas dimensiones básicas, es costumbre expresar otras dimensiones de interés en diferentes campos de la física y la ingeniería como derivadas de estas dimensiones básicas, estas dimensiones derivadas pueden surgir por la existencia de una ley física que relacione ambas magnitudes o por definiciones que resulten operativamente útiles \parencites{himme}{volker}.

Entonces un sistema de unidades toma una serie de dimensiones básicas a partir de las cuales se eligen arbitrariamente unidades básicas para realizar mediciones sobre las dimensiones básicas. Generalmente en física se suelen elegir como dimensiones básicas las siguientes dimensiones; longitud \textbf{L}\nomenclature{\textbf{L}}{Longitud}, tiempo \textbf{T}\nomenclature{\textbf{T}}{Tiempo}, masa \textbf{M}\nomenclature{\textbf{M}}{Masa}, temperatura \(\bm{\Theta}\)\nomenclature{\(\bm{\Theta}\)}{Temperatura}, corriente eléctrica \textbf{I}\nomenclature{\textbf{I}}{Corriente eléctrica}, cantidad de materia \textbf{N}\nomenclature{\textbf{N}}{Cantidad de materia}, y la intensidad luminosa \textbf{J}\nomenclature{\textbf{J}}{Intensidad luminosa}, estas ultimas son la base del denominado Sistema Internacional de Unidades (SI), las unidades básicas para cada dimensión, así como los símbolos usados en el SI se presentan en la tabla \ref{tab:unidades_si} \parencite{volker}.

\begin{longtable}[htbp!]{lclc}
    \caption{Resumen de las dimensiones y unidades usadas en el SI.} \label{tab:unidades_si} \tabularnewline
    \toprule
    Dimension base        & Símbolo de la dimensión & Unidad base & Símbolo de la unidad \\ \midrule(lr)
    Longitud              & \textbf{L}              & Metro       & \si{\metre}          \\
    Masa                  & \textbf{M}              & Kilogramo   & \si{\kilogram}       \\
    Tiempo                & \textbf{T}              & Segundo     & \si{\second}         \\
    Corriente eléctrica   & \textbf{I}              & Ampere      & \si{\ampere}         \\
    Cantidad de sustancia & \textbf{N}              & Mol         & \si{\mole}           \\
    Intensidad luminosa   & \textbf{J}              & Candela     & \si{\candela}        \\
    Temperatura           & \(\bm{\Theta}\)         & Kelvin      & \si{\kelvin}         \\
    \bottomrule
    \caption*{Adaptada de \cite{doran}}
\end{longtable}

\subsubsection{Sistemas termodinámicos}

Un sistema termodinámico es un cierto volumen en el espacio o una cantidad finita de materia implicados en el proceso termodinámico bajo estudio, este sistema debe de estar delimitado por una superficie o frontera, que puede ser real, imaginaria o una combinación de estas, fija o móvil, que separa al sistema del resto del volumen o materia existente fuera del sistema, a este entorno circundante al sistema se le denomina de varias maneras; medio ambiente, entorno o alrededores. Como se puede observar en la figura \ref{fig:exp_sistema} el sistema no tiene porque tener una forma regular, basta con que tenga una manera de diferenciarlo de los alrededores.

\begin{figure}[htbp]
    \centering
    \includegraphics{images/sistema1.pdf}
    \caption{Ejemplo de un sistema termodinámico y su frontera.}
    \label{fig:exp_sistema}
\end{figure}

Los sistemas pueden clasificarse de varias maneras: Un \emph{sistema cerrado} es aquel en el que no hay flujo de masa a través de su frontera, es decir aquel sistema con una masa invariable, pero puede intercambiar energía con los alrededores por medio de sus fronteras, por el contrario un \emph{sistema abierto} es aquel donde existe un flujo a través de las fronteras del sistema, por lo que la materia al interior del sistema no permanece invariable, y puede intercambiar energía con los alrededores, por ultimo existe el \emph{sistema aislado} cuyas fronteras impiden completamente el intercambio de energía y masa hacia los alrededores, no existen sistemas completamente aislados pero se puede utilizar el concepto de sistema aislado para aproximar diferentes sistemas.

A los sistemas cerrados también se les puede denominar \emph{masa de control}, y a los sistemas abiertos como \emph{volumen de control}\footnote{Es importante aclarar que un volumen de control puede tener fronteras móviles, y por tanto cambiar su volumen durante el proceso, aunque este ultimo caso es poco usual.}, y a su vez a la frontera de un sistema abierto se le denomina \emph{superficie de control}. En este punto es preciso hacer una aclaración sobre el concepto de masa invariable del párrafo anterior, en los sistemas abiertos, aun cuando la magnitud de la masa al interior del sistema abierto puede permanecer constante, no se considera como una masa invariable ya que al existir entrada y salida de materia, las partículas presentes en un instante de tiempo no son las mismas que en cualquier otro instante de tiempo, razón por la cual se considera que la masa ha cambiado aun cuando en magnitud puedan ser iguales \parencites{clavell}{faires}{wark}.

En la figura \ref{sfig:cerrado} se puede observar un sistema conformado por la masa presente al interior de un dispositivo cilindro émbolo, en este caso la frontera es en real en su totalidad, dado que la frontera coincide con los limites físicos del recipiente y la cara interior del émbolo. Mientras que en la figura \ref{sfig:abierto} el volumen de control conformado por una sección de la tubería posee una frontera mixta, ya que una parte de la frontera es la superficie interior de la tubería, en tanto que la otra parte es una elección arbitraría para delimitar la sección en la tubería.

\begin{figure}[htbp]
    \centering
    \begin{subcaptionblock}{0.4\linewidth}
        \centering
        \includegraphics{images/sistemacerradoejem}
        \caption{Sistema cerrado en un cilindro émbolo}
        \label{sfig:cerrado}
    \end{subcaptionblock}
    \begin{subcaptionblock}{0.4\linewidth}
        \centering
        \includegraphics{images/sistemaabiertoejem}
        \caption{Sistema abierto en una tubería}
        \label{sfig:abierto}
    \end{subcaptionblock}
    \caption{Representación de dos sistemas termodinámicos}
\end{figure}

\subsubsection{Propiedades}

Para que la termodinámica pueda estudiar los cambios de energía en un sistema es necesario poder describir el sistema en un momento determinado, a estas variables que describen a un sistema se les conoce como \emph{propiedades}, las propiedades son características macroscópicas de un sistema entre las que se incluyen la presión, temperatura, densidad, y volumen especifico a las que se les puede asignar un valor numérico independientemente de la historia del sistema. Con frecuencia una propiedad es directamente mensurable, o puede ser obtenida como el producto de otras propiedades, nótese como las propiedades llevan asociadas una dimensión y por tanto unidades, de modo que su valor depende del sistema de unidades utilizado en la medición \parencites{moranshapiro}{faires}{wark}.




\clearpage
\renewcommand{\nomname}{Nomenclatura del capítlo}
\printnomenclature

\printbibliography[title=Bibliografía del capítulo]

\end{document}