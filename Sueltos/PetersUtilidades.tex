\documentclass[10pt,letterpaper]{article}
\usepackage{polyglossia} %configuración de idioma
\setmainlanguage{spanish} %elegir idioma español
\usepackage{fontspec} %paquete para cargar fuentes
\setsansfont{cmunb}[Extension=.otf,
UprightFont=*mr,
ItalicFont=*mo,
BoldFont=*sr, % semibold
BoldItalicFont=*so, % semibold oblique
NFSSFamily=cmbr
]
\usepackage{cmbright} %fuentes
\usepackage[nopatch=eqnum]{microtype} %mejorar generales al documento pdf
\usepackage{amsmath} %paquete matemáticos
\usepackage{amsfonts} %fuentes matemáticas varias
\usepackage{amssymb} %símbolos matemáticos varios
\usepackage{graphicx} %incluir gráficos
\usepackage[left=2cm, right=2cm, top=2cm, bottom=2cm]{geometry} %paquete de configuración de margenes
\usepackage[dvipsnames]{xcolor} %paquete de colores
\usepackage[backend=biber, style=apa]{biblatex} %Bibliografía APA
\DeclareLanguageMapping{spanish}{spanish-apa} %Idioma Apa
\usepackage[hidelinks]{hyperref} %Soporte para links internos y externos, hidelinks esconde los cuadros de texto del vinculo
\usepackage{float} %Agregar imágenes en cualquier lugar
\setlength{\bibitemsep}{0.3\baselineskip} %espaciado entre referencias
\setlength{\parindent}{0mm} %sangría de párrafo
\setlength{\parskip}{1.75ex} %espaciado de párrafo
\usepackage{chemformula}
\title{Demostración de una formula general para el cálculo del criterio de Peters y utilidad bruta}
\author{Daniel Hernández}
\begin{document}
\maketitle

\section{Demostración}

Considere de forma general las reacciones de síntesis a partir de un número cualquiera (\( i \)) de reactivos para formar \( j \) productos, donde solo uno de ellos puede ser recuperado para su venta, lo que se expresa en la formula \ref{f:rxn_gen}.

\begin{equation}
    \ch{\( \nu_{R_1} \) R1 + \( \dots  \)  + \( \nu_{R_i} \) R_i -> \( \nu_{P_1} \) P_1 + \( \dots  \)  + \( \nu_{P_j} \) P_j} \label{f:rxn_gen}
\end{equation}

\begin{tabular}{ccl}
    Donde:  & \ch{R_i}  & es el reactivo i-ésimo de la reacción \\
             & \ch{P_j}  & es el producto j-ésimo de la reacción  \\
             & \( \nu_{\ch{R_i}} \) & es el coeficiente estequiométrico del reactivo i-ésimo en la reacción balanceada \\
             & \( \nu_{\ch{P_j}} \) & es el coeficiente estequiométrico del producto j-ésimo en la reacción balanceada
\end{tabular}

Si se asume que unicamente es posible recuperar dinero de la venta de un producto principal, entonces se puede seleccionar un producto \ch{P_k}, tal que \( 1 \leq k \leq j \), como producto principal, los datos del resto de productos se pueden omitir, por lo que se puede reformular la reacción \ref{f:rxn_gen} cuya resultado se muestra en la reacción \ref{f:rxn_simp}.

\begin{equation}
    \ch{\( \nu_{R_1} \) R1 + \( \dots  \)  + \( \nu_{R_i} \) R_i -> \( \nu_{P_k} \) P_k} \label{f:rxn_simp}
\end{equation}

A su vez si se introduce el concepto de eficiencia (\( \eta \)) para la reacción, la reacción \ref{f:rxn_simp} se puede nuevamente ajustar como se muestra en la reacción \ref{f:rxn_n}:

\begin{equation}
    \ch{\( \nu_{R_1} \) R1 + \( \dots  \)  + \( \nu_{R_i} \) R_i -> \( \eta \nu_{P_k} \) P_k} \label{f:rxn_n}
\end{equation}

Ahora se puede definir a \ch{PM_{R_i}} como el peso molecular del reactivo i-ésimo, y a \ch{PM_{P_k}} como el peso molecular del reactivo principal, así como a \ch{x_{R_i}} como el precio del reactivo i-ésimo por unidad de masa, y a  \ch{y_{P_k}} como el precio por unidad de masa del producto de interés.

Entonces si se planea producir una determinada cantidad \ch{M} del producto de interés \ch{P_k}, para calcular los costos se puede transformar los moles de la reacción en sus equivalentes másicos mediante el peso molecular de cada compuesto, la conversión a masa se muestra en la reacción \ref{f:rxn_mass}.

\begin{equation}
    \ch{\( \nu_{R_1} \) PM_{R_1} R1 + \( \dots  \)  + \( \nu_{R_i} \) PM_{R_i} R_i -> \( \eta \nu_{P_k} \)PM_{P_k}  P_k} \label{f:rxn_mass}
\end{equation}

Para poder obtener la utilidad bruta se necesita ajustar la reacción para obtener el gasto en reactivos para producir \ch{M} del producto de interés, el ajuste se puede realizar mediante el factor \( \frac{M}{\eta \nu_{P_k} \ch{PM_{P_K}}} \), por lo que el ajuste queda de la siguiente manera:

\begin{equation}
    \frac{\ch{M}}{\eta \nu_{\ch{P_k}} \ch{PM_{P_K}}} \nu_{\ch{R_1}} \ch{PM_{R_1} R_1} + \dots + \frac{M}{\eta \nu_{\ch{P_k}} \ch{PM_{P_k}}} \nu_{\ch{R_i}} \ch{PM_{R_i} R_i}  \ch{->} \ch{M P_k} \label{f:rxn_ajus}
\end{equation}

Entonces para encontrar la utilidad bruta (\( U_b \)) según la formula \ref{eq:Ub} es necesario encontrar el gasto en reactivos y la recuperación esperada por la venta del producto.

\begin{equation}
    U_b = \text{Recuperación del producto} - \text{Costo de reactivos} \label{eq:Ub}
\end{equation}

Por lo tanto si tomamos las masas de producto y reactivo y se multiplican por su precio por unidad de masa, se puede encontrar el valor de los reactivos y los productos, por lo que:

\begin{align}
    \text{Recuperación del producto} &= \ch{M y_{Pk}} \label{eq:recu} \\
    \text{Costo de reactivos} &= \sum_{n = 1}^{i} \frac{\ch{M}}{\eta \nu_{\ch{P_k}} \ch{PM_{P_K}}} \nu_{\ch{R_n}} \ch{PM_{R_n} x_{n}} \label{eq:costo}
\end{align}

Se expreso el costo de reactivos en forma de sumatoria por comodidad, además es posible simplificar la expresión \ref{eq:costo} sacando el factor constante:

\begin{equation}
    \text{Costo de reactivos} = \frac{\ch{M}}{\eta \nu_{\ch{P_k}} \ch{PM_{P_K}}} \sum_{n = 1}^{i}  \nu_{\ch{R_n}} \ch{PM_{R_n} x_{n}} \label{eq:costo2}
\end{equation}

Al sustituir \ref{eq:recu} y \ref{eq:costo2} en \ref{eq:Ub} se tiene:

\begin{align}
    U_b &= \ch{M y_{Pk}} - \frac{\ch{M}}{\eta \nu_{\ch{P_k}} \ch{PM_{P_K}}} \sum_{n = 1}^{i}  \nu_{\ch{R_n}} \ch{PM_{R_n} x_{n}} \\
    U_b &= \ch{M} \left[ \ch{y_{Pk}} - \frac{1}{\eta \nu_{\ch{P_k}} \ch{PM_{P_K}}} \sum_{n = 1}^{i}  \nu_{\ch{R_n}} \ch{PM_{R_n} x_{n}} \right] \label{eq:Ubfinal}
\end{align}

Por lo tanto la ecuación \ref{eq:Ubfinal} proporciona una formula para calcular la utilidad bruta para la producción de \( M \) unidades de masa del producto de interés. A partir de esta definición se podría definir la utilidad bruta por unidad de masa \( u_b \) como:

\begin{equation}
    u_b = \frac{U_b}{\ch{M}} = \ch{y_{Pk}} - \frac{1}{\eta \nu_{\ch{P_k}} \ch{PM_{P_K}}} \sum_{n = 1}^{i}  \nu_{\ch{R_n}} \ch{PM_{R_n} x_{n}} \label{eq:ub}
\end{equation}

Ahora para el criterio de Peters \( \text{Pet} \) se tiene la siguiente formula

\begin{equation}
    \text{Pet} = \frac{\text{Costo de reactivos}}{\text{Recuperación del producto}} \label{eq:peters}
\end{equation}

Si se sustituye \ref{eq:recu} y \ref{eq:costo2} en \ref{eq:peters} se obtiene:

\begin{align}
    \text{Pet} &= \frac{\frac{\ch{M}}{\eta \nu_{\ch{P_k}} \ch{PM_{P_K}}} \sum_{n = 1}^{i}  \nu_{\ch{R_n}} \ch{PM_{R_n} x_{n}}}{\ch{M y_{Pk}}}
    \intertext{Al simplificar \ch{M} y reordenar la ecuación tenemos:}
    \text{Pet} &= \frac{\frac{1}{\eta \nu_{\ch{P_k}} \ch{PM_{P_K}}} \sum_{n = 1}^{i}  \nu_{\ch{R_n}} \ch{PM_{R_n} x_{n}}}{\ch{y_{Pk}}} \label{eq:petdir} \\
    \text{Pet} &= \frac{1}{\eta \nu_{\ch{P_k} \ch{PM_{P_K}}}} \cdot \frac{\sum_{n = 1}^{i}  \nu_{\ch{R_n}} \ch{PM_{R_n} x_{n}}}{\ch{y_{Pk}}}
\end{align}

Ahora la ecuación \ref{eq:petdir} se puede utilizar para calcular directamente el valor del criterio de Peters, que como se puede apreciar es independiente de la cantidad \ch{M} que se quiere producir, también es posible hallar la relación entre el valor de \ch{Pet} y el de \( U_b \) o \( u_b \) mediante algunos arreglos a las formulas, tomando como base la ecuación \ref{eq:ub} se tiene:

\begin{align}
    u_b &=  \ch{y_{P_k}} - \frac{1}{\eta \nu_{\ch{P_k}} \ch{PM_{P_K}}} \sum_{n = 1}^{i}  \nu_{\ch{R_n}} \ch{PM_{R_n} x_{n}} \notag
    \intertext{Al dividir entre \ch{y_{Pk}}}
    \frac{u_b}{\ch{y_{PK}}} &= \frac{1}{ \ch{y_{Pk}}} \left[ \ch{y_{Pk}} - \frac{1}{\eta \nu_{\ch{P_k}} \ch{PM_{P_K}}} \sum_{n = 1}^{i}  \nu_{\ch{R_n}} \ch{PM_{R_n} x_{n}} \right] \\
    \frac{u_b}{\ch{y_{PK}}} &= 1 - \frac{\frac{1}{\eta \nu_{\ch{P_k}} \ch{PM_{P_K}}} \sum_{n = 1}^{i}  \nu_{\ch{R_n}} \ch{PM_{R_n} x_{n}}}{\ch{y_{Pk}}}
    \intertext{Usando la ecuación \ref{eq:petdir}}
    u_b &= (1 - \text{Pet}) \cdot  \ch{y_{PK}}
    \intertext{Finalmente reemplazando la definición dada para \( u_b \) }
    U_b &= (1- \text{Pet}) \cdot \ch{y_{PK}} \cdot \text{M}
\end{align}



\end{document}
