\usepackage{polyglossia} %configuración de idioma
\setmainlanguage{spanish} %elegir idioma español
\usepackage{fontspec} %paquete para cargar fuentes
\setsansfont{cmunb}[Extension=.otf,
UprightFont=*mr,
ItalicFont=*mo,
BoldFont=*sr, % semibold
BoldItalicFont=*so, % semibold oblique
NFSSFamily=cmbr]
\usepackage{cmbright} %fuentes
\usepackage{microtype} %mejorar generales al documento pdf
\usepackage{amsmath} %paquete matemáticos
\usepackage{amsfonts} %fuentes matemáticas varias
\usepackage{amssymb} %símbolos matemáticos varios
\usepackage{graphicx} %incluir gráficos
\usepackage[left=2cm, right=2cm, top=2cm, bottom=2cm]{geometry} %paquete de configuración de margenes
\usepackage[dvipsnames]{xcolor} %paquete de colores
\usepackage[backend=biber, style=apa]{biblatex} %Bibliografía APA
\DeclareLanguageMapping{spanish}{spanish-apa} %Idioma Apa
%\addbibresource{Libros/Biblioteca.bib} %Archivo .bib
\setlength{\bibitemsep}{0.5\baselineskip} %espaciado entre referencias
\setlength{\parindent}{0mm} %sangria de parrafo
\setlength{\parskip}{1.5pt}
\usepackage{epstopdf} %Agregar archivos svg
\usepackage{float} %Agregar imágenes en cualquier lugar
