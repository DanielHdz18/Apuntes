\documentclass[../master.tex]{subfiles}
\begin{document}

    \section{Preliminares}

    \subsection{Recolección y análisis de datos de velocidad}

    Una ecuación cinética caracteriza a la velocidad de reacción y su expresión suele provenir de consideraciones teóricas o de el ajuste de curvas de datos experimentales, actualmente los modelos predictivos son ineficientes y la mayoría de las ecuaciones cinéticas se obtienen a partir de datos experimentales \parencite{levenspiel1}.

    La determinación de la ecuación cinética suele realizarse mediante un procedimiento de dos etapa; primero se determina la variación de la velocidad con la concentración a temperatura constante, y después la variación de los coeficientes cinéticos para obtener la ecuación cinética completa \parencite{levenspiel1}.

    

\end{document}
